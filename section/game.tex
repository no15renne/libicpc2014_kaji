\section{game / ゲーム}
\subsection{Nim}
石の山が$n$個あって、それぞれ$a_i$個の石を含んでいる。\\
2人が交互に空でない山を1つ選び、そこから1個以上の石をとり、最後の石をとったら勝ちである。\\
このとき先攻、後攻のどちらが勝つか。

以下が成立する。\\
$a_1~XOR~a_2~XOR~\ldots~XOR~a_n~\neq~0 \rightarrow$ 勝ちの状態\\
$a_1~XOR~a_2~XOR~\ldots~XOR~a_n~=~0 \rightarrow$ 負けの状態\\
これよりXORを計算し、0でなければ先攻の勝ち、0ならば後攻の勝ちとなる。\\
負けの状態からは勝ちの状態にしかいけないこと、勝ちの状態からは常に負けの状態が作れることから証明できる。\\
以下は先攻がAlice,後攻がBobの例
\lstinputlisting[caption=Nim]{src/game/nim.cpp}

\subsection{Grundy数}
Grundy数を用いることで多くのゲームをNimに帰着できる。\\
「今の状態から一手でいける状態のGrundy数に含まれていない最小の非負の整数が、今の状態のGrundy数」\\
このGrundy数はNimの1つの山と似た性質を持つ。


