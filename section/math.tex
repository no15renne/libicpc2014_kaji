\section{math / 数学}

\subsection{シンプソンの公式(合成)}
$\int_a^b f(x) \mathrm{d}x$を$N$点で分割し近似する.最大誤差は,$-\frac{h^4}{180}(b-a)f^{(4)}(\xi)$ 程度.
\lstinputlisting[caption=Simpson's rule]{src/math/simpson.cpp}

\subsection{ユークリッド互除法}
ユークリッド互除法
$O(\log \max(a,b))$
\lstinputlisting[caption=Eucledean algorithm]{src/math/gcd.cpp}

\subsection{拡張ユークリッド互除法}
拡張ユークリッド互除法
$ax + by = gcd(a,b)$となる$x,y$を求める
\lstinputlisting[caption=Extension Eucledean algorithm]{src/math/extgcd.cpp}

\subsection{Prime}
素数判定、約数の列挙、素因数分解
\lstinputlisting[caption=Prime]{src/math/prime.cpp}

\subsection{エラトステネスの篩}
エラトステネスの篩
$O(n \log\log n)$
\lstinputlisting[caption=Eratosthenes]{src/math/eratosthenes.cpp}

\subsection{行列の操作}
\lstinputlisting[caption=Matrix]{src/math/matrix.cpp}

\subsection{繰り返し二乗法}
累乗を高速に求める
$O(\log n)$
\lstinputlisting[caption=Exponential]{src/math/exponential.cpp}

\subsection{$m$項漸化式}
繰り返し二乗法を利用すれば,どちらも$O(m^3 \log k)$
\begin{itemize}
	\item 定数項がない場合: $a_{n+m} = b_{m-1}a_{n+m-1} + \dots + b_0a_n$
	\[
		\left[\begin{array}{c} a_{n+m} \\ a_{n+m-1} \\ \vdots \\ a_{n+1} \end{array} \right]
		= \left[\begin{array}{cccc}
			b_{m-1} & \dots & b_1 & b_0 \\
			1 &  &  &  \\
			& \ddots &  &  \\
			&  & 1 &
		\end{array} \right]
		\left[\begin{array}{c} a_{n+m-1} \\ a_{n+m-2} \\ \vdots \\ a_n \end{array} \right]
	\]
	\item 定数項がある場合: $a_{n+m} = b_{m-1}a_{n+m-1} + \dots + b_0a_n + c$
	\[
		\left[\begin{array}{c} a_{n+m} \\ a_{n+m-1} \\ \vdots \\ a_{n+1} \\ 1 \end{array} \right]
		= \left[\begin{array}{ccccc}
			b_{m-1} & \dots & b_1 & b_0 & c \\
			1 &  &  &  & \\
			& \ddots &  &  & \\
			&  & 1 &  & \\
			&  &  &  & 1 \\
		\end{array} \right]
		\left[\begin{array}{c} a_{n+m-1} \\ a_{n+m-2} \\ \vdots \\ a_n \\ 1 \end{array} \right]
	\]
\end{itemize}

\subsection{合同式の性質}
\begin{itemize}
	\item $a \equiv a \pmod{p}$
	\item $a \equiv b \pmod{p} \Rightarrow b \equiv a \pmod{p}$
	\item $a \equiv b \pmod{p},~ b \equiv c \pmod{p} \Rightarrow a \equiv c \pmod{p}$
	\item $a \equiv b \pmod{p},~ c \equiv d \pmod{p} \Rightarrow a+c \equiv b+d \pmod{p}$
	\item $a \equiv b \pmod{p},~ c \equiv d \pmod{p} \Rightarrow a-c \equiv b-d \pmod{p}$
	\item $a \equiv b \pmod{p},~ c \equiv d \pmod{p} \Rightarrow ac \equiv bd \pmod{p}$
	\item $a \equiv b \pmod{p} \Rightarrow a^k \equiv b^k \pmod{p}$
	\item $ka \equiv kb \pmod{p} \Rightarrow a \equiv b \pmod{\frac{p}{\gcd(k,p)}}$
\end{itemize}

\subsection{フェルマーの小定理}
$p$を素数,$a$を$p$で割り切れない自然数とすると,
\[
	a^{p-1} \equiv 1 \pmod{p} \mbox{または}~ a^{p} \equiv a \pmod{p}.
\]
